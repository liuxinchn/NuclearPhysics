\chapter{原子核的组成}

\section{历程} 

\begin{enumerate}
    \item 电子的发现 
    \item 原子核的发现 $\rightarrow$卢瑟福用$\alpha$轰击金箔,有八千分之一的几率被反射。 $\Rightarrow$ 正电荷和原子质量几种在原子中心。
    \item 质子的发现 $\rightarrow$用$\alpha$粒子轰击$^{14}N$,发现了质子。
    \item 早期原子核组成的想法及其碰到的困难
    \item 中子的发现 $\alpha$ + $^{9}Be$ $\Rightarrow$ $^{12}C$ + n  ,用n再打石蜡能出来质子,根据能量,推测出不是射线,是中子(质量跟质子差不多)  【中子穿透性强,寿命短~14.81min】
\end{enumerate}

$_{z}^{A}X_{n}$,z是质子数,A是核子数,N为中子数

\section{质子和中子的性质对比} 

\begin{enumerate}
    \item 中子比质子质量大一点点  938  939
    \item 质子寿命非常长,$10^21$year,中子很短,14.81min。
    \item 随之探测技术发展,中子带一点点负电$(-0.4\pm 1.1)*10^{-31}e$,可以认为不带电。
\end{enumerate}

\section{亚核子自由度}

电子散射:用电子轰击原子核,来推测原子核内部结构。

质子和中子电荷分布示意图:表明质子和中子并不是最微观的粒子。

\section{夸克}

质子和中子由夸克组成,总共有6种夸克:上夸克(up)、下夸克(down)、顶夸克(top)、底夸克(bottom)、粲夸克(charm)、奇异夸克(strange)。其中up/top/charm带三分之二的正电荷,dowm/bottom/strange带三分之一的负电荷。

质子和中子是费米子,夸克也是费米子。

质子由三个夸克组成,uud(两个up,一个down)

中子由三个夸克组成,ddu(两个down,一个up)

\section{夸克禁闭}

带色的粒子不能单独存在,夸克总是和别的夸克禁闭在一起而形成色中性的强子。

强子中的夸克疯狂的交换胶子进行强作用,他们存在于由胶子组成的色场中:当胶子场获得足够能量时,就会折断成一对夸克-反夸克。

夸克禁闭问题至今还没有完全解决清楚。

遗留问题:核子质量大约是电子1800倍,而核子由三个夸克组成,则电子不是由夸克组成,猜想还有比夸克更小的粒子。

\section{轻子}

总共6种轻子:e(电子)、$\tau$($\tau$子)、$\mu$ ($\mu$子)、Ve(电子中微子)、V$\tau$($\tau$子中微子)、V$\tau$($\mu$子中微子)

粒子物理标准模型:6种夸克+6种轻子+传递力的粒子

\clearpage
