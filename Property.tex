\chapter{原子核的静态性质}

原子核物理学是研究原子核性质、结构和转化的科学。

\begin{itemize}
    \item 静态性质包括:原子核的半径、质量、自旋、磁矩、电四极矩、宇称、统计性、同位旋
    \item 动态性质包括:衰变寿命、分质比
\end{itemize}

\section{原子核的电荷} 

\subsection{电荷} 
 
原子核带正电,电子带负电,正负电荷相抵消,原子对外不带电。

\subsection{测量电荷数Z}

使用莫塞莱定律,测量原子特征X射线的波长$\lambda$ ,根据$c=v\lambda$,求出频率$v$,然后根据$\sqrt{v}=AZ-B$(其中A、B为常数),求出Z。

\textbf{X射线的由来:}对于原子来说,核外电子在不同轨道运动,轨道之间有固定的能量差,给核外电子一个能量,电子会被激发到其他轨道上,在它退激发的过程中,就会释放出电磁波,这个电磁波就是X射线。

\subsection{核素图}

纵坐标表示质子数Z,横坐标表示中子数N。

稳定核素($\beta$稳定线上的)大约300个,实验室合成出来的3000个,理论上预言6000-8000个。

\section{原子核的质量}

\subsection{质量}

碳单位:原子的质量太小,为了方便计算,将$^{12}C$ 质量的 $\frac{1}{12}$作为1u(实际质量值为$1.661e^{-27}kg$),其他原子的相对质量就是其原子质量与1u的比值,如$^{16}O$的相对原子质量为16($2.657e^{-26}/1.661e^{-27}=16$)。
 
质能方程:$E = Mc^2$,1u = 931.448MeV(1质量数对应的能量为931.448兆电子伏)。

\begin{itemize}
    \item 质子静止质量:$938.280MeV/c^2$
    \item 中子静止质量:$939.573MeV/c^2$
    \item 电子静止质量:$511.003keV/c^2$
\end{itemize}

\subsection{测量质量}

\begin{enumerate}
    \item 用质谱仪(针对稳定核好用)
    \item 飞行时间法
    \item 用核反应精确测定
\end{enumerate}

\vspace{1.2em}

测出原子核质量之后,可以知道原子核能量,根据原子核能量=质子能量+中子能量+结合能(释放出来),可以算出结合能。知道了结合能,就能知道给原子核多少能量,它可以分开。结合能的意义就是原子核结合的紧密程度。

\section{原子核的半径}

\subsection{半径}

半径$10^{-15}m$,也可以说1fm

\subsection{测量半径}

可以~电子~或~质子/中子~去打原子核,根据反应截面,得出半径。

\begin{itemize}
    \item 用电子打时,依据时电磁相互作用,测的是电荷分布半径。
    \item 用~质子/中子~去打时,依据时强相互作用(核力),测的是核力作用半径。
\end{itemize}

核力作用半径更大一些。可以理解为:电荷分布半径是依据质子与电子的电磁作用,核力作用半径是依据核子之间的相互作用,加上了中子,表现的更大一些。

半径与质量数的1/3成正比 【$R\varpropto A^{1/3}$】

\begin{description}
    \item[核力作用半径] $R \approx (1.40\pm 0.10)A^{1/3}fm$ 
    \item[电荷分布半径] $R \approx (1.20\pm 0.30)A^{1/3}fm$ 
\end{description}

根据$V=\frac{4}{3} \pi R^{3}$,则$R\varpropto A$,可推出,原子核具有不可压缩性(后来被推翻,发现了晕核)。

根据$\rho = \frac{A}{V}$,可以算出原子核的密度,密度特别大(每立方厘米有亿吨重)。(中子星-全是核子)

\subsection{总结}

\begin{enumerate}
    \item 有两种半径:核力作用半径、电荷分布半径。核力作用半径稍微大一点。
    \item 半径与质量数的三分之一成正比,体积与质量数成正比。
\end{enumerate}

\section{原子核的自旋}

\subsection{自旋}

电子的自旋是$\frac{1}{2}$ ,自旋跟角动量对应。

电子的角动量是自旋角动量和轨道角动量耦合得到的。

拓展到原子核:

核子的角动量=核子的自旋+轨道角动量。

原子核的角动量=所有核子的角动量的耦合。

\subsection{测量自旋}

测量电子的自旋用的是原子光谱的精细结构。

在原子光谱的精细结构中,轨道分为 s p d f g h i k,对应的轨道角动量为0 1 2 3 4 5 6 7

j、l和s分别是电子的总角动量、轨道角动量和自旋角动量量子数。j=l+s,l+s-1,\dots,$\left\lvert l-s \right\rvert $

钠发黄光,是钠电子在s p轨道,角动量分别是0 1,电子自旋角动量是$\frac{1}{2}$,这样就有$P_{\frac{3}{2}}$ $P_{\frac{1}{2}}$ $S_{\frac{1}{2}}$ 

测量核的基态自旋的方法是利用原子光谱的超精细结构。

F、i和j分别是原子的总角动量、原子核的总角动量和电子的总角动量。F=i+j,i+j-1,\dots,$\left\lvert i-j \right\rvert $

原子核的总角动量很小,在精细结构中,可以将其忽略。在超精细结构中,不能忽略其影响。但在钠发黄光的能谱中,它只影响了s轨道。

\textbf{怎么测量核自旋}

\begin{itemize}
    \item 如果$I\leq J$,就有2I+1个F值(即能级分裂为2I+1个能级),数原子光谱中超精细结构的数目即可求得I。
    \item 如果$I\geq J$,那么能级分裂为2J+1个,显然无法由数亚能级数目来确定I。F=I+J, I+J-1, … 亚谱线的相邻间距满足$\Delta E_1 : \Delta E_2 : \Delta E_3 : \dots = (I+J) : (I+J-1) : (I+J-2) : \dots $
\end{itemize}

\begin{description}
    \item[偶偶核] Z/N都是偶数  基态自旋是0
    \item[奇A核] Z/N一个是偶数 基态自旋是半整数
    \item[奇奇核] Z/N都是奇数  基态自旋是整数
\end{description}

质子存在两两抱对的现象,中子也是。稳定核一般是偶偶核,或者奇A核。偶偶核更稳定。

\section{原子核的磁矩}

\subsection{磁矩}

$\mu _1 = g_1 \mu _N I$,其中$\mu _1$为磁矩,$g_1$为g因子,$\mu _N$为核磁子,$I$为自旋。

\subsection{测量磁矩}

核磁共振。

偶偶核的磁矩为0,使用奇A核,如医疗上核磁共振是探究$^1_1H_0$在人体的分布情况,核磁共振没有辐射。

\section{原子核的电四极矩}

原子核大多都不是标准的球型,有许多是椭球形(还有三轴形变等...),用电四极矩来描述椭球形的原子核。

将椭球形放到坐标轴中,两个相等的边为a、b,另一个与前两者不相等的为c。a=b<c为长椭球,a=b>c为扁椭球。

$Q=\frac{2}{5}Z(c^2-a^2)$

\begin{itemize}
    \item 当c=a时,Q=0,即球形核的电四极矩为零。
    \item 当c>a时,Q>0,即长椭球形原子核具有正的电四极矩。
    \item 当c<a时,Q<0,即扁椭球形原子核具有负的电四极矩。
\end{itemize}

\section{原子核的宇称}

\section{原子核的统计性质}

\section{原子核的同位旋}

\clearpage