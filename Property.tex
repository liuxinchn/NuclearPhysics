\chapter{原子核的静态性质}

原子核物理学是研究原子核性质、结构和转化的科学。

\begin{itemize}
    \item 静态性质包括:原子核的半径、质量、自旋、磁矩、电四极矩、宇称、统计性、同位旋
    \item 动态性质包括:衰变寿命、分质比
\end{itemize}

\section{原子核的电荷} 

\subsection{电荷} 
 
原子核带正电,电子带负电,正负电荷相抵消,原子对外不带电。

\subsection{测量电荷数Z}

使用莫塞莱定律,测量原子特征X射线的波长$\lambda$ ,根据$c=v\lambda$,求出频率$v$,然后根据$\sqrt{v}=AZ-B$(其中A、B为常数),求出Z。

\textbf{X射线的由来:}对于原子来说,核外电子在不同轨道运动,轨道之间有固定的能量差,给核外电子一个能量,电子会被激发到其他轨道上,在它退激发的过程中,就会释放出电磁波,这个电磁波就是X射线。

\subsection{核素图}

纵坐标表示质子数Z,横坐标表示中子数N。

稳定核素($\beta$稳定线上的)大约300个,实验室合成出来的3000个,理论上预言6000-8000个。

\section{原子核的质量}

\subsection{质量}

碳单位:原子的质量太小,为了方便计算,将$^{12}C$ 质量的 $\frac{1}{12}$作为1u(实际质量值为$1.661e^{-27}kg$),其他原子的相对质量就是其原子质量与1u的比值,如$^{16}O$的相对原子质量为16($2.657e^{-26}/1.661e^{-27}=16$)。
 
质能方程:$E = Mc^2$,1u = 931.448MeV(1质量数对应的能量为931.448兆电子伏)。

\begin{itemize}
    \item 质子静止质量:$938.280MeV/c^2$
    \item 中子静止质量:$939.573MeV/c^2$
    \item 质子静止质量:$511.003keV/c^2$
\end{itemize}

\subsection{测量质量}

\begin{enumerate}
    \item 用质谱仪(针对稳定核好用)
    \item 飞行时间法
    \item 用核反应精确测定
\end{enumerate}

\vspace{1.2em}

测出原子核质量之后,可以知道原子核能量,根据原子核能量=质子能量+中子能量+结合能(释放出来),可以算出结合能。知道了结合能,就能知道给原子核多少能量,它可以分开。结合能的意义就是原子核结合的紧密程度。

\section{核半径}

半径$10^{-15}m$,也可以说1fm

可以~电子~或~质子/中子~去打。

用电子打时,依据时电磁相互作用,测的是电荷半径;用~质子/中子~去打时,依据时强相互作用(核力),测的是核力作用半径。
 
结果核力作用半径更大一些。可以理解为加上质子/中子,表现的更大一些。

体积与质量数成正比(除了晕核),推出,原子核具有不可压缩性(后来被推翻)。

中子星-全是核子,密度特别大。

\section{原子核的自旋}

电子的自旋是$\frac{1}{2}$ ,自旋跟角动量对应。

电子的角动量是自旋角动量和轨道角动量耦合得到的。

拓展到原子核,所有核子自旋和轨道角动量的矢量和,构成了原子核的总角动量。

核子的角动量=核子的自旋+轨道角动量。

原子核的角动量=所有核子的角动量的耦合。

测量核的基态自旋的方法是利用原子光谱的超精细结构。

\section{原子核的磁矩}

\section{原子核的电四极矩}

\section{原子核的宇称}

\section{原子核的统计性质}

\section{原子核的同位旋}

\clearpage