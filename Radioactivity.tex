\chapter{衰变规律}

\section{放射性衰变的基本规律}

\subsection{放射性的一般现象}

\begin{itemize}
    \item 激发态的不稳定性———$\gamma$衰变 [电磁相互作用]
    \item 核基态的不稳定性———核转变($\beta$或$\alpha$衰变)[弱相互作用和电磁相互作用]
    \item 共振态衰变———粒子发射 [强相互作用或核作用]
\end{itemize}

\textbf{ps:}$\alpha$或$\beta$衰变之后,并不一定处于子核的基态上,大多数情况处于子核的激发态上,再通过$\gamma$衰变回到子核基态。

\textbf{放射性:}原子核自发地放射各种射线的现象。

\textbf{$\beta$衰变:}

$^A_ZX \rightarrow ^A_{Z+1}Y + e^-$ (n$\rightarrow$p)

$^A_ZX \rightarrow ^A_{Z-1}Y + e^+$ (p$\rightarrow$n)

$^A_ZX + e^- \rightarrow ^A_{Z-1}Y$ (p$\rightarrow$n)

\textbf{天然放射系:}锕系、钍系、铀系。其他的还有天然放射性元素,如:$^{209}_{83}Bi$。

\subsection{放射性衰变的指数衰减规律}

$N=N_0e^{-\lambda t}$

$\lambda $称为衰变常数,是在单位时间内每个原子核的衰变概率,它的量纲是时间的倒数。它的意思是衰变概率。

\begin{enumerate}
    \item 各个粒子的行为相互独立。
    \item 过程发生的概率与“历史”无关。
    \item 在极小的时空间隔里,过程发生的概率正比于该间隔。
\end{enumerate}

\vspace{1.2em}

\textbf{半衰期:}$T_{1/2}=\frac{ln2}{\lambda} $

\textbf{平均寿命:}$\tau =\frac{1}{\lambda}$

接下来看第12周的视频,同时多看看第11周的ppt,因为里边有很多数学公式,没看懂。