\chapter{绪论}

\section{物理的发展}

\begin{description}
    \item[经典物理] 19世纪末之前
    \item[近现代物理] 19世纪末20世纪初 $\sim $ 现在  
\end{description}

\section{近现代物理的研究方向}

\begin{enumerate}
    \item 宇观。星球 $\rightarrow$ 星系 $\rightarrow$ 宇宙
    \item 微观。原子物理与粒子物理。
    \item 各个层次之间的联系。大的物质由小的物质组成,小物质之间的联系、大物质与小物质之间的联系就是研究的方向。
\end{enumerate}

\section{核科学的发展}

\begin{enumerate}
    \item 1895年,伦琴发现X射线,核科学的开端。(原理是核外电子在不同轨道之间跃迁,会释放出能量,这就是X射线)
    \item 1896年,贝克勒尔发现了铀的天然放射性。(铀盐无论是否在太阳下曝晒,都能使胶片感光)
    \item 1897年,汤姆逊发现电子。(做实验过程中发现有一种粒子在磁场中发生偏转,偏转的方向标明带负电)
    \item 1898年,居里夫妇分离出放射性的钋和镭。(在贝克勒尔的基础上,发现铀矿石的放射性比铀盐中的放射性强度要强,然后分离出钋和镭)
    \item 1898年,卢瑟福发现$\alpha$、$\beta$射线。
    \item 1900年,维拉德发现$\gamma$射线。
    \item 1905年,爱因斯坦提出相对论。
    \item 1909年,卢瑟福验证了$\alpha$粒子就是氦原子核。($\alpha$粒子打到玻璃管上,隔一段时间去检测,出现了氦,先猜测$\alpha$粒子是氦原子核,从玻璃管上拿到电子,形成氦原子,后又验证了)
    \item 1911年,卢瑟福用$\alpha$粒子轰击金箔发现了原子核。(八千分之一的概率弹回来)
    \item 1914年,莫塞莱用X射线测定原子核的电荷。
    \item 1919年,卢瑟福首次实现人工核反应,发现质子。(用$\alpha$粒子打$^{14}N$,打出了一种粒子,命名为质子)
    \item 1932年,查德威克发现中子。
    \item 1938年,Hahn和strassman发现重核裂变。
    \item 1939年,建立了裂变的液滴模型。
    \item 1942年,费米等实现受控的链式核反应。
    \item 1945年,第一颗原子弹爆炸。
\end{enumerate}

\clearpage