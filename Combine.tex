\chapter{原子核的结合能与液滴模型}

\section{原子核的结合能}

\subsection{质量与能量的一般关系}

\textbf{质能方程:}$E=mc^2$

相对论还给出,以速度数值$v$运动的粒子为例,其中$m_0$是粒子的静止质量。当粒子的速度v增大时,它的质量$m$也随之增大。

$$ m = \frac{m_0}{\sqrt{ 1 - (\frac{v}{c})^2}} $$

总能量$E$ = 静止质量$m_0c^2$ + 动能$E_k$,即$E=m_0c^2 + E_k$

对于光子来说,静止质量为零,但有动能。光子的总能量等于它的动能。

\textbf{质量亏损:}组成某一原子核的所有核子质量和与该原子核质量的差称为原子核的质量亏损。

\textbf{广义质量亏损:}体系变化前后静止质量之差。

\textbf{原子核的结合能:}自由核子组成原子核时放出的能量或原子核分解为自由核子时吸收的能量,都叫原子核的结合能,它是原子核整体稳定性的度量。

例如,一个中子和一个质子结合成氘核时,要放出2.22兆电子伏的能量,这个能量以γ光子的形式辐射出去。

\textbf{比结合能:}原子核平均每个核子的结合能又称为比结合能。

比结合能表示,若把原子核拆成自由核子,平均对于每个核子所要做的功。比结合能ε的大小,可以用以标志原子核结合的松紧程度。越大的原子核结合的越紧;越小的原子核结合的较松。比结合能的物理意义是将原子核拆散成自由核子时,外界对每个核子所做的最小的平均功。

\section{液滴模型}

